\openshmem provides ISO \Cstd language bindings. Any implementation that
provides \Cstd bindings can claim conformance to the specification. The
\openshmem header file \HEADER{shmem.h} for \Cstd must contain only the
interfaces and constant names defined in this specification.

\openshmem \acp{API} can be implemented as either routines or macros. However,
implementing the interfaces using macros is strongly discouraged as this could
severely limit the use of external profiling tools and high-level compiler
optimizations. An \openshmem program should avoid defining routine names,
variables, or identifiers with the prefix \shmemprefix, \shmemprefixC, or with
\openshmem \ac{API} names.

All \openshmem extension \acp{API} that are not part of this specification must
be defined in the \HEADER{shmemx.h} include file for language bindings. This
header file must exist, even if no extensions are provided. Any extensions
shall use the \FUNC{shmemx\_} prefix for all routine, variable, and constant
names. An exception to this restriction is made for \openshmem extensions
that add support for additional types to existing \openshmem routines.
